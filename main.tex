If and when a paper on astroquery is written, we'll use this article (and its
related github repo: https://github.com/adamginsburg/astroquery-paper) as the
basis.  However, it is not even a work in progress yet.

\section{Introduction}
Sharing data is a critical component of astronomical research.  Astronomy
has historically been a leading field in data sharing, motivated at least
in part by questions that cannot be answered with single instruments.
In the past few decades, blind surveys have played a huge role in advancing our
understanding of the universe.

Data sharing has taken on a variety of forms.  The most prominent are the major
observatory archives: MAST, NOAO, ESO, IPAC, CADC, CDS (hosting Vizier and
SIMBAD), NRAO, CXC, HEASARC, ESA are the main
organizations hosting raw and processed data from ground and space based
telescopes.  These data archive also serve as the primary means for serving data
to users when the data are taken in queue mode, i.e., when the data are taken
while the observer is not on-site.

In addition to observatories and telescopes, individual surveys often share
their full data sets.  In some cases, these data sets are shared via the
observatory that acquired them - for example, the all-sky data acquired with
Planck, WMAP, and COBE delivered a variety of data products as part of the
mission.  Other surveys, particularly ground-based surveys, serve their own
data.  Examples include SDSS, UKIDSS, ...

Individual teams and small groups will often share their data.
These services do not follow any particular standard and can be widely
varied in the type and amount of data shared.

Finally, there are other data types relevant to astronomy that are not
served by the typical astronomical databases.  Examples include molecular
and atomic properties.

Astroquery arose from a desire to access these databases from the command line
in a scriptable fashion.  Script-based data access provides astronomers with
the ability to make reproducible analysis scripts in which the data are
acquired and processed into scientifically relevant results with minimal
overhead.

The Virtual Observatory has some overlap with astroquery, but it does not
provide the simple tools many astronomers find necessary in day-to-day work.
Additionally, many of the services noted above do not support VO standards or
protocols and are therefore inaccessible to the VO.


\section{The Software}
Astroquery consists of a collection of modules that mostly share a similar interface,
but are meant to be used independently.  They are primarily based on a common framework
that uses the python \texttt{requests} package to perform HTTP requests to communicate
with web services.

For new development, there is a \texttt{template}  that lays out the basic
framework of any new module.  All modules are based on having a single core
\texttt{class} that will have some number of \texttt{query\_*} methods.
The most common query methods are \texttt{query\_region}, which usually provide
a ``cone search'' functionality, i.e., they search for data within a circularly
symmetric region projected on the sky.

An example using the SIMBAD interface is shown below (see
\url{http://astroquery.readthedocs.io/en/latest/simbad/simbad.html}):
\begin{lstlisting}

    from astroquery.simbad import Simbad
    result_table = Simbad.query_region("m81")

\end{lstlisting}
In this example, \texttt{Simbad} is an instance of the
\texttt{astroquery.simbad.SimbadClass} class.
The returned result, stored in the variable \texttt{result\_table},

\section{The development model}
Astroquery has received contributions from 53 people as of June 2017.
While the primary maintenance burden is shouldered by 2-3 people at any given time,
most individual modules have been implemented independently by interested volunteers.
Additionally, astroquery has received support from the Google Summer of Code
program, with three students (coauthors Madhura Parikh, Simon Liedtke, and Ayush Yadav)
from 2013-2017.


