\documentclass{article}
\usepackage[utf8]{inputenc}
\usepackage{url}
\usepackage{listings}
\begin{document}


\begin{abstract}
Astroquery is a collection of tools for requesting data from databases hosted
on the internet, particularly those with web pages but without formal
application program interfaces (APIs).  These tools are based on the Python
requests module, which is used to make HTTP requests, and astropy, which
provides most of the data parsing functionality.  Astroquery has received
significant contributions from the broader astronomical community, including
several significant contributions from telescope archives.
\end{abstract}

associated repository:
https://github.com/adamginsburg/astroquery-paper

\section{Introduction}
Sharing data is a critical component of astronomical research.  Astronomy
has historically been a leading field in data sharing, motivated at least
in part by questions that cannot be answered with single instruments.
In the past few decades, blind surveys have played a huge role in advancing our
understanding of the universe.

Data sharing has taken on a variety of forms.  The most prominent are the major
observatory archives: MAST, NOAO, ESO, IPAC, CADC, CDS (hosting Vizier and
SIMBAD), NRAO, CXC, HEASARC, and ESA are the main
organizations hosting raw and processed data from ground and space based
telescopes.  These data archives also serve as the primary means for serving data
to users when the data are taken in queue mode, i.e., when the data are taken
while the observer is not on-site.

In addition to observatories and telescopes, individual surveys often share
their full data sets.  In some cases, these data sets are shared via the
observatory that acquired them - for example, the all-sky data acquired with
Planck, WMAP, and COBE delivered a variety of data products as part of the
mission.  Other surveys, particularly ground-based surveys, serve their own
data.  Examples include SDSS, UKIDSS, and likely many more in the near future.

Individual teams and small groups will often share their data.
These services do not follow any particular standard and can be widely
varied in the type and amount of data shared.  Sometimes these data
are shared via the archive systems (e.g., IRSA at IPAC hosts many
individual survey data sets), while others use their own web hosting
systems (e.g., MAGPIS).

Finally, there are other data types relevant to astronomy that are not
served by the typical astronomical databases.  Examples include molecular
and atomic properties, such as those provided by Splatalogue and NIST.

Astroquery arose from a desire to access these databases from the command line
in a scriptable fashion.  Script-based data access provides astronomers with
the ability to make reproducible analysis scripts in which the data are
acquired and processed into scientifically relevant results with minimal
overhead.

In this paper, we provide an overview of the astroquery package.
Section \ref{sec:software} describes the basic layout of the software and
the shared API concept underlying all modules.  Section \ref{sec:development}
describes the development model.

% The Virtual Observatory has some overlap with astroquery, but it does not
% provide the simple tools many astronomers find necessary in day-to-day work.
% Additionally, many of the services noted above do not support VO standards or
% protocols and are therefore inaccessible to the VO.


\section{The Software}
\label{sec:software}
Astroquery consists of a collection of modules that mostly share a similar interface,
but are meant to be used independently.  They are primarily based on a common framework
that uses the Python \texttt{requests} package to perform HTTP requests to communicate
with web services.

For new development, there is a \texttt{template}  that lays out the basic
framework of any new module.  All modules are based on having a single core
\texttt{class} that will have some number of \texttt{query\_*} methods.
The most common query methods are \texttt{query\_region}, which usually provide
a ``cone search'' functionality, i.e., they search for data within a circularly
symmetric region projected on the sky.

An example using the SIMBAD interface is shown below (see
\url{http://astroquery.readthedocs.io/en/latest/simbad/simbad.html}):
\begin{lstlisting}

    from astroquery.simbad import Simbad
    result_table = Simbad.query_region("m81")

\end{lstlisting}
In this example, \texttt{Simbad} is an instance of the
\texttt{astroquery.simbad.SimbadClass} class.
The returned result, stored in the variable \texttt{result\_table},
is an astropy table.

While there is a common suggested API described in the \texttt{template} module,
individual packages are not \emph{required} to support this API because, for
some, it is not possible.  For example, the atomic and molecular databases refer
to physical data that is not related to positions on the sky and therefore
their astroquery modules cannot include \texttt{query\_region} methods.

\subsection{The API}
The common API has a few features defined in the \texttt{template} module.
Each service is expected to provide the following interfaces, assuming they are
applicable:

\begin{itemize}
    \item \texttt{query\_region} - A function that accepts an astropy
        \texttt{SkyCoord} object representing a point on the sky plus a
        specification of the radius around which to search.
        The returned object is an astropy \texttt{Table}.
    \item \texttt{query\_object} - A function that accepts the name of an
        object.  This method relies on the service to resolve the object name.
        The returned object is an astropy \texttt{Table}.
    \item \texttt{get\_images} - For services that provide image data, this
        function accepts an astropy \texttt{SkyCoord} and a radius to search for data
        that cover the specified target. The returned object should be a list
        of \texttt{astropy.io.fits.HDUList} objects.
\end{itemize}

Beyond these basic functions, there is a series of functions with the same
names, but with the additional suffix \texttt{\_async}.  The
\texttt{query\_*\_async} functions return a \texttt{requests.Response} object
from the accessed website or a list of such objects, providing developers with
the ability to access the data in a stream or access only the response
metadata.  The \texttt{get\_images\_async} method returns
\texttt{FileContainer} objects that similarly provide `lazy' access to the
data, but specifically for FITS files.  Developers need only implement
these \texttt{\_async} functions because a wrapper tool exists to convert
\texttt{\_async} functions into their corresponding non-asynchronous versions.

\subsection{Testing}
Astroquery testing is somewhat different from most other packages in the Python
ecosystem.  While the tests are based on the astropy package-template and use
pytest to run and check the outputs, the astroquery tests are split into
\emph{remote} and \emph{non-remote}.  The remote tests exactly replicate what a user
would enter at the command line, but they are dependent on the stability of the
remote services.  Historically, we have found that it is quite rare for all of
the astroquery-supported services to be accessible simultaneously.

We therefore require that each module provide some tests that do not rely on
having an internet connection.  These tests rely on \emph{monkeypatching} to
replace the remote requests, instead using locally available files to test the
query mechanisms and the data parsers.  Monkeypatching in the context of
\texttt{pytest} results in code that is generally more difficult to understand
than typical Python code, but a set of tests independent of the remote services
is necessary.

The non-remote tests are run as part of the continuous integration for the
project with each commit.  The remote tests are run as part of a
regularly-scheduled \texttt{cron} job.  Running the remote tests infrequently
also helps reduce the burden on the remote services.

\subsection{Other utilities}
There are several general-use utilities implemented as part of astroquery, such
as a bulk FITS file downloader and renamer and a download progressbar.  There
is also a schema system implemented to allow user-side parameter validation.
So far, at the time of writing, this tool is only implemented in the ESO and
Vizier modules, but it could be expanded to other modules to reduce the number
of doomed-to-fail queries sent through astroquery.

\section{The development model}
\label{sec:development}
Astroquery has received contributions from 53 people as of June 2017.
While the primary maintenance burden is shouldered by 2-3 people at any given time,
most individual modules have been implemented independently by interested volunteers.

Some contributions have come as direct institutional support.  The ESA GAIA and
ESASky modules were provided by developers working for ESA.  Meanwhile,
the MAST and VO Cone Search query tools were added by developers at STSCI,
with the latter moved over from \texttt{astropy.vo}.

Astroquery has received support from the Google Summer of Code
program, with two students (co-authors Madhura Parikh and Simon Liedtke)
from 2013-2017.

Anyone can contribute to astroquery.  The maintainers are committed to helping
developers make new modules that meet the requirements of astroquery.

\section{Documentation and References}
Several authors have independently described how to use various astroquery
modules, which is a helpful practice we encourage.

\url{https://www.cosmosim.org/cms/news/cosmosim-package-for-astroquery/}
\url{https://arxiv.org/abs/1408.7026}
\end{document}
