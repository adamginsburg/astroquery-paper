If and when a paper on astroquery is written, we'll use this article (and its
related github repo: https://github.com/adamginsburg/astroquery-paper) as the
basis.  However, it is not even a work in progress yet.

\section{Introduction}
Sharing data is a critical component of astronomical research.  Astronomy
has historically been a leading field in data sharing, motivated at least
in part by questions that cannot be answered with single instruments.
In the past few decades, blind surveys have played a huge role in advancing our
understanding of the universe.

Data sharing has taken on a variety of forms.  The most prominent are the major
observatory archives: MAST, NOAO, ESO, IPAC, CADC, CDS, NRAO are the main
organizations hosting raw and processed data from ground and space based
telescopes.  These data archive also serve as the primary means for serving data
to users when the data are taken in queue mode, i.e., when the data are taken
while the observer is not on-site.

